\documentclass[10pt]{article}
\usepackage{amsmath, amssymb, setspace, geometry}
\geometry{margin=0.75in}
\setstretch{1.15}

\begin{document}

% \begin{center}
%     \Large\textbf{STAT 5200 Project Proposal} \\[6pt]
%     \large\textbf{Tests of Asset Pricing with Time-Varying Factor Loads (Replication and Critique)}
% \end{center}

\section*{Problem / Research Question}
I plan to replicate Tests of Asset Pricing with Time-Varying Factor Loads by Galvão, Montes-Rojas, and Olmo (2019), who develop new statistical tests for evaluating linear asset-pricing models when factor loadings vary over time. I came across this paper on the JAE website and was very curious about it because it claimed to reject traditional models such as the CAPM and the Fama--French three- and five-factor models---that is, they do not correctly explain the cross-section of expected returns when betas are time-varying. 

% Their study proposes hypothesis tests for assessing whether risk premia are homogeneous across assets and whether intercepts are jointly zero, which I hope to replicate.

\section*{Dataset}
The replication data is available on the JAE website. The dataset is drawn from Kenneth French's data library and CRSP datasets. I plan on extending the dataset up to 2024 by accessing CRSP through WRDS' Python API.


\section*{Approach}
Following the paper, I will estimate regressions of the form $
r_{i,t+1}^e = \alpha_i + \beta_{i,t}'\lambda_i + \nu_{i,t+1}$
where $r_{i,t+1}^e$ denotes excess returns and $\beta_{i,t}$ are time-varying factor loadings. I will use ordinary least squares (OLS) to estimate asset-specific intercepts and risk prices, and then implement pooled or panel-based estimators to test for homogeneity across assets. I plan to compute sandwich standard errors to account for heteroskedasticity and cross-sectional dependence.

The authors rely on large-sample theory for inference. I am thinking of applying bootstrap inference to approximate the sampling distribution of the test statistics under finite samples, as discussed in class when studying resampling methods.

\section*{Extensions and Critique}
Besides extending the dataset and trying out bootstrap inference, I had a few questions when reading the paper about their methodology and econometric assumptions:
\begin{itemize}
    \item Is the conditional expectation of returns on factors necessarily linear? Could a nonlinear or state-dependent relationship between returns and factors alter the conclusions?
    \item The paper assumes $E[\nu_{i,t+1}|X_{it}] = 0$. Does this necessarily hold, especially if there are unobserved macro or liquidity factors that are not captured by the model?
    \item The asymptotic properties are derived under large $N$, large $T$, and large $m$ (frequency of observations), but in practice $N = 47$, and $m$ is bounded by daily observations. Do the asymptotic approximations hold in such finite samples? This motivates the use of bootstrap inference as an alternative.
    \item Is there omitted variable bias if the regression includes at most five factors, as in the Fama--French five-factor model? Important risk components might not be captured by these observable factors.
    \item The realized betas are estimated with error, introducing measurement-error-type bias. If betas are estimated from returns that depend on factor realizations, could there be endogeneity or attenuation bias in the estimated $\lambda_i$?
\end{itemize}

\end{document}